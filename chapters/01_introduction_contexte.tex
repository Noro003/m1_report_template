% =============================================================================
% CHAPITRE 1 : INTRODUCTION ET CONTEXTE
% =============================================================================

\chapter{Introduction et contexte}
\label{chap:introduction}

% Note pour l'étudiant : Remplacez les sections \lipsum par votre contenu réel


% \section{Présentation de l'Institut des Sciences Moléculaires d'Orsay}

% L'Institut des Sciences Moléculaires d'Orsay (ISMO) a été créé en 2010 en association avec le CNRS et l'Université Paris-Saclay. Il résulte de la fusion de trois laboratoires de la ville d'Orsay : le Laboratoire de Photophysique Moléculaire [LPPM (UPR3361)], le Laboratoire des Collisions Atomiques et Moléculaires [LCAM (UMR8625)] et le Laboratoire d'Interaction du rayonnement X avec la Matière [LIXAM (UMR8624)].

% L'ISMO rassemble 81 chercheurs ou enseignants-chercheurs permanents, 47 doctorants et 8 post-doctorants, répartis en 6 équipes scientifiques. Ils bénéficient de l'expertise de 39 personnels techniques et administratifs (IT et BIATSS), incluant 4 CDD et 2 apprentis. Depuis 2018, les membres de l'institut ont emménagé au bâtiment 520 dans le quartier du Belvédère, au sein du Centre de Physique Matière-Rayonnement.

% Les six équipes de recherche sont :
% \begin{itemize}
% \item \textbf{Dynamiques et interactions : Rayonnement, Atomes, Molécules}, qui travaille autour de l'étude de processus fondamentaux et de comportements dynamiques dans l'interaction rayonnement-matière
% \item \textbf{Nanomédecine et Biophotonique (NanoBio)}, qui exerce une activité interdisciplinaire de pointe à l'interface chimie-physique-biologie
% \item \textbf{Nanophysique et Surfaces}, qui étudie les propriétés optiques, électriques, structurelles et dynamiques des atomes, molécules et nano-objets
% \item \textbf{Structure et dynamique des systèmes complexes isolés}, caractérisée par une forte interdisciplinarité et une ouverture vers les problèmes à implication chimique ou biologique
% \item \textbf{Surfaces, Interfaces, Molecules \& 2D Materials}, dont l'activité porte sur l'étude expérimentale des propriétés physiques et chimiques des surfaces
% \item \textbf{Systèmes Moléculaires, Astrophysique et Environnement}, qui s'intéresse à l'étude des processus fondamentaux en chimie-physique, en particulier dans les domaines de la physique moléculaire et de la physique des agrégats
% \end{itemize}

% Au cours de ce stage, j'ai travaillé au sein de l'équipe \textbf{Systèmes Moléculaires, Astrophysique et Environnement} sous l'encadrement d'Olivier Pirali. Cette équipe développe des approches expérimentales et théoriques pour comprendre les processus moléculaires fondamentaux, avec des applications particulières en astrophysique et en physique de l'environnement. Les thématiques de recherche incluent la spectroscopie moléculaire haute résolution, l'étude des processus photophysiques et photochimiques, ainsi que le développement d'instrumentations avancées pour la caractérisation de systèmes moléculaires complexes.

% L'ISMO constitue un environnement de recherche stimulant où se côtoient des approches expérimentales de pointe et des développements théoriques, offrant un cadre idéal pour des projets interdisciplinaires à l'interface entre physique, chimie et astrophysique.







\section{Présentation de l'Institut des Sciences Moléculaires d'Orsay}

L'Institut des Sciences Moléculaires d'Orsay (ISMO) a été créé en 2010 en association avec le CNRS et l'Université Paris-Saclay, par la fusion de trois laboratoires d'Orsay : le Laboratoire de Photophysique Moléculaire (LPPM), le Laboratoire des Collisions Atomiques et Moléculaires (LCAM) et le Laboratoire d'Interaction du rayonnement X avec la Matière (LIXAM).

L'ISMO rassemble 81 chercheurs permanents, 47 doctorants et 8 post-doctorants. Ils sont soutenus par l'expertise de 39 personnels techniques et administratifs. L'institut est situé au sein du Centre de Physique Matière-Rayonnement.

Les six équipes de recherche de l'institut sont :

\begin{itemize}
    \item \textbf{Dynamiques et interactions : Rayonnement, Atomes, Molécules}, étudiant les processus fondamentaux de l'interaction rayonnement-matière.
    
    \item \textbf{Nanomédecine et Biophotonique (NanoBio)}, avec une activité interdisciplinaire à l'interface chimie-physique-biologie.
    
    \item \textbf{Nanophysique et Surfaces}, qui étudie les propriétés des atomes, molécules et nano-objets.
    
    \item \textbf{Structure et dynamique des systèmes complexes isolés}, caractérisée par une forte interdisciplinarité tournée vers les problèmes chimiques ou biologiques.
    
    \item \textbf{Surfaces, Interfaces, Molecules \& 2D Materials}, dont l'activité porte sur l'étude expérimentale des propriétés physico-chimiques des surfaces.
    
    \item \textbf{Systèmes Moléculaires, Astrophysique et Environnement}, qui s'intéresse aux processus fondamentaux en chimie-physique pour l'astrophysique et l'environnement.
\end{itemize}

Ce stage a été effectué au sein de l'équipe \textbf{Systèmes Moléculaires, Astrophysique et Environnement} sous l'encadrement d'Olivier Pirali. Cette équipe développe des approches expérimentales et théoriques pour comprendre les processus moléculaires fondamentaux, via la spectroscopie moléculaire haute résolution et le développement d'instrumentations avancées.

L'ISMO constitue ainsi un environnement de recherche stimulant, où se côtoient des approches expérimentales de pointe et des développements théoriques, offrant un cadre idéal pour des projets à l'interface entre physique, chimie et astrophysique.

\section*{1.1 Contexte scientifique de la spectroscopie THz}

Le domaine spectral s'étendant de 0,1 à 10 térahertz (THz) occupe une position unique et stratégique au sein du spectre électromagnétique. Située à la jonction entre l'électronique hyperfréquence et l'optique infrarouge, cette région a longtemps été qualifiée de « fossé térahertz » en raison des difficultés technologiques à générer et détecter efficacement un rayonnement cohérent. Cependant, les avancées récentes, notamment dans le domaine de la photonique, ont transformé ce fossé en une nouvelle frontière scientifique. L'importance de ce domaine réside dans le fait que les énergies des photons THz coïncident précisément avec de nombreuses excitations de basse énergie qui gouvernent la structure et la dynamique de la matière, comme les transitions rotationnelles des molécules légères, les modes de vibration de grande amplitude (torsions, inversions), ou encore les vibrations de réseau (phonons) dans les solides. La spectroscopie THz offre ainsi un accès direct à des « empreintes digitales » spectrales uniques, permettant l'identification et la caractérisation non ambiguës d'espèces chimiques et de matériaux. Le défi contemporain n'est donc plus simplement de produire un rayonnement THz, mais de le faire avec une précision, une résolution et une traçabilité métrologique extrêmes.

Les impératifs scientifiques qui motivent le développement d'instruments THz de haute précision sont principalement issus de l'astrophysique et de la physique fondamentale. En astrophysique, le rayonnement THz est une sonde irremplaçable de l'Univers « froid », où se déroulent les processus de formation des étoiles et des planètes. L'interprétation rigoureuse des spectres obtenus par les grands observatoires (Herschel, ALMA) dépend de manière critique de la disponibilité de fréquences de repos mesurées en laboratoire avec une très haute exactitude. Parallèlement, en physique fondamentale, la spectroscopie moléculaire à très haute résolution dans le domaine THz constitue un banc d'essai pour sonder les lois de la nature, notamment en recherchant une éventuelle variation temporelle des constantes fondamentales. De tels tests exigent des mesures de laboratoire d'une précision de l'ordre du kilohertz (kHz) et une résolution sub-Doppler pour s'affranchir de l'élargissement thermique des raies.

Face à ces exigences, la génération par photomélange en onde continue (CW) s'impose comme une solution pertinente. Le principe consiste à illuminer un photodétecteur ultrarapide (le photomélangeur) avec le rayonnement combiné de deux lasers continus dont la différence de fréquence, $\nu_{\text{THz}} = |\nu_{1} - \nu_{2}|$, se situe dans le domaine THz. Cette approche photonique présente des avantages décisifs : une résolution spectrale exceptionnelle, limitée uniquement par la largeur de raie des lasers sources ; une large accordabilité continue ; et une détection cohérente qui permet de mesurer l'amplitude et la phase du champ THz, rendant les acquisitions plus rapides et plus robustes en permettant de déterminer la fonction diélectrique complexe des matériaux sans recourir à des pièces mécaniques mobiles \cite{Roggenbuck2010}.

Ce projet s'appuie sur cette technologie en utilisant des composants développés pour les télécommunications optiques à 1,5 µm, notamment les lasers à rétroaction distribuée (DFB) intégrés en boîtiers compacts et fiables \cite{Stanze2010}. Si la plage d'accord d'une seule paire de ces lasers est limitée, une solution de pointe consiste à utiliser un système de trois lasers DFB ou plus. En commutant entre différentes paires, il est aujourd'hui possible d'assurer une couverture spectrale continue sur plusieurs térahertz, comme l'ont démontré des systèmes atteignant 2,75 THz \cite{Deninger2015}.

L'objectif de ce travail est donc de développer un spectromètre THz par photomélange, visant à terme une couverture de 0,3 à 5 THz, dont la performance ultime résidera dans sa chaîne de traçabilité métrologique. Pour atteindre la précision du kHz requise, l'architecture envisagée s'appuiera sur un peigne de fréquences optiques pour mesurer et asservir les fréquences des lasers DFB. Afin de garantir une exactitude absolue, ce peigne sera lui-même stabilisé par une référence de fréquence distribuée par le réseau fibré REFIMEVE, reliant ainsi l'instrument aux étalons primaires. En d'autres termes, ce système ne fait pas que générer des ondes THz : il produit une fréquence connue avec une précision absolue. Ce travail relève donc de la métrologie (la science de la mesure) et peut ainsi permettre de réaliser des expériences scientifiques fondamentales. 

% \section{Objectifs du stage}

% \subsection{Objectif principal}

% L'objectif principal de ce stage est le développement d'un spectromètre THz sub-Doppler par photomélange couvrant la gamme \thzrange. Ce système doit intégrer les dernières avancées en métrologie fréquentielle et automatisation pour atteindre les performances requises par les applications visées.

% \subsection{Objectifs spécifiques}

% \begin{enumerate}
%     \item \textbf{\objectiveone} \\
%     Concevoir et assembler un système spectroscopique basé sur le battement de lasers DFB, capable de couvrir la gamme de fréquences THz ciblée avec la résolution requise.

%     \item \textbf{\objectivetwo} \\
%     Intégrer le système dans l'infrastructure métrologique de l'ISMO, notamment le réseau \REFIMEVE{} et le peigne de fréquence \DFCcore{} pour assurer la traçabilité métrologique.

%     \item \textbf{\objectivethree} \\
%     Développer les logiciels de contrôle et d'acquisition permettant une utilisation autonome du système par les utilisateurs finaux.

%     \item \textbf{\objectivefour} \\
%     Concevoir et implémenter des algorithmes d'optimisation des performances, notamment pour le contrôle thermique des lasers.

%     \item \textbf{\objectivefive} \\
%     Évaluer les performances du système par des mesures spectroscopiques de référence et identifier les limitations pour les développements futurs.
% \end{enumerate}

% \subsection{Enjeux techniques}

% Les principaux défis techniques à relever incluent :
% \begin{itemize}
%     \item \textbf{Stabilité fréquentielle} : maintenir la cohérence spectrale sur de longues durées
%     \item \textbf{Optimisation de puissance} : maximiser le signal THz généré
%     \item \textbf{Contrôle thermique} : gérer la dérive des lasers avec la température
%     \item \textbf{Métrologie} : assurer la traçabilité et la précision des mesures de fréquence
% \end{itemize}


% \section{Plan du rapport}

% Ce rapport retrace chronologiquement le développement du spectromètre, depuis la conception théorique jusqu'aux premiers résultats expérimentaux. Chaque chapitre développe un aspect spécifique du projet, en mettant l'accent sur les innovations apportées et les enseignements tirés.

% Le \cref{chap:protocoles} détaille l'architecture système et les protocoles expérimentaux développés. Le \cref{chap:resultats} présente les performances obtenues et les compare aux spécifications. Le \cref{chap:analyse} replace les résultats dans leur contexte scientifique et prospectif. Enfin, le \cref{chap:conclusion} synthétise les acquis du stage et leurs implications pour la suite du projet.