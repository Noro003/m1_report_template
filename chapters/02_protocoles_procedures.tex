% =============================================================================
% CHAPITRE 2 : PROTOCOLES ET PROCÉDURES
% =============================================================================

\chapter{Protocoles et procédures}
\label{chap:protocoles}

Ce chapitre présente l'architecture technique du spectromètre THz développé, les protocoles expérimentaux mis en place, et les algorithmes utilisées. 
\section{Principe du photomélange}

\subsection{Théorie du battement optique}

Le photomélange repose sur la génération d'un rayonnement THz par battement de deux lasers dans un photomélangeur. Considérons deux ondes optiques de fréquences \freq{1} et \freq{2} :

\begin{align}
E_1(t) &= A_1 \cos(2\pi f_1 t + \phi_1) \\
E_2(t) &= A_2 \cos(2\pi f_2 t + \phi_2)
\end{align}

Le champ total incident sur le photomélangeur s'écrit :
\begin{equation}
E_{\text{total}}(t) = E_1(t) + E_2(t)
\end{equation}

L'intensité détectée par le photomélangeur, proportionnelle à $E_{\text{total}}^2(t)$, contient un terme de battement à la fréquence différence :

\begin{beatequation}
\freqbeat = |f_1 - f_2|
\end{beatequation}

Cette fréquence de battement, située dans le domaine THz, constitue le signal d'intérêt pour la spectroscopie.

\subsection{Photomélangeur utilisé}

Le photomélangeur choisi est une photodiode p-i-n InGaAs (modèle PCA-FD-1550-100-TX-1) optimisée pour la gamme \SI{1550}{\nano\meter}. Ses caractéristiques principales sont :
\begin{itemize}
    \item Bande passante : DC à \SI{100}{\GHz}
    \item Responsivité : \SI{0.8}{\ampere\per\watt} à \SI{1550}{\nano\meter}
    \item Puissance optique maximale : \SI{100}{\milli\watt}
\end{itemize}

\subsection{Modélisation de la puissance THz}

La puissance THz générée suit approximativement une loi de puissance dans la gamme d'intérêt :

\begin{powerequation}
\powerTHz(\freqbeat) = P_0 \left(\frac{\freqbeat}{f_0}\right)^{-\alpha}
\end{powerequation}

où $P_0$ est la puissance de référence à $f_0 = \SI{0.5}{\THz}$ et $\alpha \approx 1.3$ l'exposant mesuré expérimentalement.


\section{Architecture du système}

\subsection{Vue d'ensemble}

Le spectromètre THz développé s'articule autour de quatre sous-systèmes principaux :
\begin{enumerate}
    \item \textbf{Sources laser} : deux lasers DFB accordables
    \item \textbf{Génération THz} : photomélangeur et optique associée
    \item \textbf{Détection} : bolomètre et amplificateur lock-in
    \item \textbf{Métrologie} : peigne de fréquence et réseau \REFIMEVE{}
\end{enumerate}

\begin{figure}[ht]
    \centering
    % Remplacez par votre schéma système
    \includegraphics[width=0.9\textwidth]{figures/schema_systeme.pdf}
    \caption{Schéma de principe du spectromètre THz par photomélange. Les deux lasers DFB génèrent le battement THz détecté par le bolomètre après interaction avec l'échantillon gazeux.}
    \label{fig:schema_systeme}
\end{figure}

\subsection{Lasers DFB et contrôle}

\subsubsection{Spécifications des lasers}

Deux lasers DFB (Distributed Feedback) constituent les sources optiques :

\textbf{Laser 1 :}
\begin{itemize}
    \item Gamme de fréquence : \SIrange{191.702}{192.163}{\THz}
    \item Plage d'accordabilité : \SI{461.4}{\GHz}
    \item Contrôleur : \Arroyo{} (\comport{4})
\end{itemize}

\textbf{Laser 2 :}
\begin{itemize}
    \item Gamme de fréquence : \SIrange{192.721}{193.156}{\THz}
    \item Plage d'accordabilité : \SI{435.2}{\GHz}
    \item Contrôleur : \Arroyo{} (\comport{3})
\end{itemize}

\subsubsection{Gamme THz résultante}

La combinaison des deux lasers permet de couvrir la gamme THz :
\begin{equation}
\freqbeat \in [\SI{0.558}{\THz}, \SI{1.455}{\THz}]
\end{equation}

soit une plage totale de \SI{896.6}{\GHz}, proche de la limite théorique de \SI{1.5}{\THz}.

\subsubsection{Contrôle thermique et en courant}

Chaque laser est piloté par son contrôleur \Arroyo{} via :
\begin{itemize}
    \item \textbf{Courant de polarisation} : \SIrange{100}{200}{\milli\ampere}
    \item \textbf{Température TEC} : \SIrange{15}{35}{\celsius}
\end{itemize}

La fréquence d'émission suit les relations empiriques :

\begin{dfbequation}
f(I, T) = f_0 + \alpha_I (I - I_0) + \alpha_T (T - T_0)
\end{dfbequation}

avec les coefficients d'accordabilité $\alpha_I$ et $\alpha_T$ fournis par le constructeur.

\subsection{Système de détection}

\subsubsection{Bolomètre}

La détection THz utilise un bolomètre hélium refroidi présentant :
\begin{itemize}
    \item Bande passante : \SIrange{0.1}{5}{\THz}
    \item Sensibilité : $\SI{e4}{\volt\per\watt}$ typique
    \item Constante de temps : \SI{1}{\milli\second}
\end{itemize}

\subsubsection{Amplificateur lock-in}

Le signal du bolomètre est traité par un amplificateur lock-in \SR{830} (\gpibport{7}) configuré en mode :
\begin{itemize}
    \item Fréquence de référence : \SI{1}{\kilo\hertz}
    \item Constante de temps : \SI{100}{\milli\second}
    \item Sensibilité : adaptée automatiquement
\end{itemize}

La modulation est appliquée par hachage mécanique du faisceau THz.

\section{Métrologie fréquentielle}

\subsection{Peigne de fréquence optique}

Le système intègre un peigne de fréquence \Toptica{} \DFCcore{} pour la métrologie. Ce peigne génère un spectre régulier de modes séparés de \frep{} :

\begin{equation}
f_n = n \cdot \frep + \fceo
\end{equation}

où $n$ est l'ordre du mode et \fceo{} l'offset carrier-envelope.

\subsection{Réseau REFIMEVE}

Le réseau métrologique \REFIMEVE{} fournit deux signaux de référence :
\begin{itemize}
    \item \textbf{Signal radiofréquence} : \SI{10}{\GHz} ultra-stable
    \item \textbf{Signal optique infrarouge} : \SI{1542}{\nano\meter}
\end{itemize}

La stabilité relative atteint $10^{-17}$ à $10^{-18}$ sur les échelles de temps pertinentes.

\section{Automatisation et contrôle logiciel}

\subsection{Architecture logicielle}

Le système de contrôle s'articule autour de deux workflows indépendants :

\begin{enumerate}
    \item \textbf{Génération de paramètres} : calcul des points de mesure optimaux
    \item \textbf{Exécution des scans} : automatisation des acquisitions
\end{enumerate}

\subsection{Workflow 1 : Génération de paramètres}

Le module \texttt{dfb\_parameter\_calculator\_gui.py} implémente une interface graphique permettant de :
\begin{itemize}
    \item Définir les contraintes de puissance et température
    \item Calculer les paramètres optimaux (courant, température)
    \item Exporter les fichiers CSV de paramètres
\end{itemize}

\subsection{Workflow 2 : Exécution automatisée}

Le module \texttt{advanced\_tec\_csv\_scanner\_fixed.py} réalise l'acquisition automatisée :
\begin{itemize}
    \item Lecture des paramètres CSV
    \item Contrôle des instruments via VISA/série
    \item Acquisition synchronisée des données
    \item Sauvegarde automatique avec horodatage
\end{itemize}

\section{Innovation : Optimisation TEC}

\subsection{Problématique identifiée}

L'analyse des temps d'exécution a révélé que les commandes TEC (Temperature Electric Cooler) constituent le facteur limitant principal :
\begin{itemize}
    \item Temps par commande TEC : \SIrange{60}{90}{\second}
    \item Temps par commande courant : \SI{5}{\second}
    \item Rapport : 12 à 18 fois plus lent
\end{itemize}

\subsection{Algorithme d'optimisation développé}

L'algorithme développé analyse les paramètres de scan pour éviter les commandes TEC redondantes :

\begin{algorithm}[ht]
\caption{Optimisation TEC intelligente}
\begin{algorithmic}[1]
\For{each scan point $i$}
    \State $\Delta T = |T_i - T_{i-1}|$
    \If{$\Delta T < \epsilon_T$}
        \State Skip TEC command
        \State Set laser current only
    \Else
        \State Send TEC setpoint
        \State Wait for stabilization
        \State Set laser current
    \EndIf
    \State Acquire data
\EndFor
\end{algorithmic}
\end{algorithm}

Le seuil $\epsilon_T = \SI{0.001}{\celsius}$ a été optimisé expérimentalement.

\subsection{Monitoring passif de température}

Pour les points sans changement TEC, un monitoring passif vérifie la stabilité :

\begin{itemize}
    \item Période d'observation : \SI{25}{\second}
    \item Critère de stabilité : $\sigma_T < \SI{0.02}{\celsius}$
    \item Validation automatique de la convergence
\end{itemize}

\section{Protocoles expérimentaux}

\subsection{Préparation du système}

Avant chaque session de mesure :
\begin{enumerate}
    \item Vérification de la stabilisation thermique (\SI{30}{\minute} minimum)
    \item Calibration des détecteurs
    \item Optimisation des phases lock-in
    \item Vérification de la métrologie fréquentielle
\end{enumerate}

\subsection{Procédure de scan standard}

Un scan typique suit la séquence :
\begin{enumerate}
    \item Chargement des paramètres CSV
    \item Configuration des instruments
    \item Exécution du scan avec optimisation TEC
    \item Sauvegarde automatique des données
    \item Validation de la qualité des mesures
\end{enumerate}

\subsection{Gestion des erreurs et sécurités}

Le système intègre plusieurs niveaux de sécurité :
\begin{itemize}
    \item Surveillance continue des températures
    \item Limitation des puissances optiques
    \item Détection de perte de communication
    \item Arrêt d'urgence automatique
\end{itemize}

\section{Caractérisation du système}

\subsection{Étalonnage en fréquence}

L'étalonnage s'appuie sur les données constructeur des lasers DFB, validées ponctuellement par comparaison avec le peigne de fréquence. La précision attendue est de l'ordre du \SI{1}{\MHz}.

\subsection{Étalonnage en puissance}

Un puissancemètre \texttt{PM100D} calibré permet la caractérisation de la chaîne optique. Les mesures de puissance THz restent relatives, le bolomètre n'étant pas étalonné en valeur absolue.

Cette architecture système constitue la base expérimentale pour les mesures spectroscopiques présentées au chapitre suivant.