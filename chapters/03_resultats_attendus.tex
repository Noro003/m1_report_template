% =============================================================================
% CHAPITRE 3 : RÉSULTATS VS ESCOMPTÉS
% =============================================================================

\chapter{Résultats vs escomptés}
\label{chap:resultats}

Ce chapitre présente les performances obtenues avec le spectromètre THz développé et les confronte aux spécifications initiales. Il détaille les résultats expérimentaux, quantifie les gains apportés par les innovations techniques, et établit un diagnostic des limitations rencontrées.

\section{Performances générales du système}

\subsection{Gamme de fréquences accessible}

Le système développé couvre effectivement la gamme THz spécifiée :

\begin{keyresult}
\textbf{Gamme THz réalisée :} \SIrange{0.558}{1.455}{\THz} \\
\textbf{Plage totale :} \SI{896.6}{\GHz} \\
\textbf{Objectif initial :} \SIrange{0.3}{5}{\THz} (partiellement atteint)
\end{keyresult}

Cette gamme représente 60\% de l'objectif initial de \SI{1.5}{\THz}, constituant déjà un résultat significatif compte tenu des limitations technologiques dans cette région spectrale.

\subsection{Stabilité et reproductibilité}

Les mesures de stabilité sur \SI{1}{\hour} montrent :
\begin{itemize}
    \item \textbf{Dérive fréquentielle} : \SIrange{15}{30}{\MHz} sans isolateurs optiques
    \item \textbf{Stabilité de puissance} : \SI{5}{\percent} RMS typique
    \item \textbf{Reproductibilité des enveloppes spectrales} : validation qualitative
\end{itemize}

\section{Innovation TEC : Quantification des gains}

\subsection{Méthodologie d'évaluation}

L'évaluation de l'optimisation TEC s'appuie sur un scan de référence de 2200 points, représentatif des mesures spectroscopiques typiques. Les métriques analysées incluent :
\begin{itemize}
    \item Temps total d'exécution
    \item Nombre de commandes TEC évitées
    \item Répartition des temps par type d'opération
\end{itemize}

\subsection{Résultats quantitatifs spectaculaires}

\begin{keyresult}
\textbf{Scan de 2200 points - Comparaison Performance :}
\begin{itemize}
    \item \textbf{Sans optimisation :} \SI{16.5}{\hour}
    \item \textbf{Avec optimisation :} \SI{11}{\hour}
    \item \textbf{Gain de temps :} 33\% (5.5h économisées)
    \item \textbf{Commandes TEC évitées :} 1911/2256 = 84.7\%
\end{itemize}
\end{keyresult}

\subsection{Analyse détaillée des temps}

Le \cref{tab:temps_operations} détaille la répartition des temps par type d'opération :

\begin{table}[ht]
    \centering
    \caption{Répartition des temps d'exécution par opération}
    \label{tab:temps_operations}
    \begin{tabular}{lcc}
        \toprule
        \textbf{Type d'opération} & \textbf{Temps typique} & \textbf{Impact relatif} \\
        \midrule
        Changement courant seul & \SI{5}{\second} & Référence \\
        Changement température & \SIrange{60}{90}{\second} & 12-18× plus lent \\
        Monitoring passif & \SI{25}{\second} & 5× plus rapide que TEC \\
        Acquisition données & \SI{2}{\second} & Négligeable \\
        \bottomrule
    \end{tabular}
\end{table}

\subsection{Validation par l'encadrant}

Cette optimisation a été explicitement validée par l'encadrant Olivier Pirali, confirmant l'intérêt scientifique et technique de cette innovation pour la communauté spectroscopique.

\section{Mesures spectroscopiques préliminaires}

\subsection{Molécule test : vapeur d'eau}

Les premiers tests spectroscopiques ont été réalisés sur la vapeur d'eau (\HdO) ambiante, molécule de référence en spectroscopie THz. Les conditions expérimentales étaient :
\begin{itemize}
    \item \textbf{Pression :} atmosphérique (\SI{1013}{\hecto\pascal})
    \item \textbf{Température :} ambiante (\SI{20}{\celsius})
    \item \textbf{Longueur d'absorption :} \SI{50}{\centi\meter}
\end{itemize}

\subsection{Résultats obtenus}

La \cref{fig:spectre_h2o} présente un spectre typique obtenu dans la gamme \SIrange{0.8}{1.2}{\THz} :

\begin{figure}[ht]
    \centering
    % Remplacez par vos données réelles
    \includegraphics[width=0.9\textwidth]{figures/spectre_h2o_exemple.pdf}
    \caption{Spectre de la vapeur d'eau dans la gamme \SIrange{0.8}{1.2}{\THz}. L'enveloppe spectrale montre la signature caractéristique des transitions rotationnelles, mais les fluctuations chaotiques empêchent l'identification de raies individuelles.}
    \label{fig:spectre_h2o}
\end{figure}

\subsection{Qualité spectrale observée}

\textbf{Points positifs :}
\begin{itemize}
    \item Détection confirmée du signal d'absorption
    \item Enveloppes spectrales reproductibles
    \item Gamme de fréquences conforme aux attentes
\end{itemize}

\textbf{Limitations identifiées :}
\begin{itemize}
    \item Fluctuations chaotiques rapides du signal
    \item Impossibilité d'identifier des raies individuelles
    \item Rapport signal/bruit non quantifiable (pas de ligne de base stable)
\end{itemize}

\section{Diagnostic technique : Rétroaction optique}

\subsection{Approche diagnostique systématique}

Face aux limitations spectrales observées, une approche diagnostique systématique a été déployée :

\begin{enumerate}
    \item \textbf{Validation logicielle} : vérification des logs et de l'exécution correcte
    \item \textbf{Analyse matérielle} : identification des sources de bruit et d'instabilité
    \item \textbf{Investigation de la rétroaction optique} : diagnostic principal
\end{enumerate}

\subsection{Identification de la limitation principale}

\begin{keyresult}
\textbf{Limitation critique identifiée : Rétroaction optique}
\begin{itemize}
    \item \textbf{Cause :} Absence d'isolateurs optiques sur les sorties DFB
    \item \textbf{Symptômes :} Spectres chaotiques, fluctuations rapides
    \item \textbf{Mécanisme :} Feedback optique $\rightarrow$ chaos spectral DFB
    \item \textbf{Impact :} Limitation principale empêchant spectres exploitables
\end{itemize}
\end{keyresult}

\subsection{Mécanisme physique}

La rétroaction optique dans les lasers DFB induit des instabilités spectrales bien documentées dans la littérature. Le feedback des éléments optiques downstream (photomélangeur, optiques) vers la cavité laser provoque :
\begin{itemize}
    \item Élargissement spectral
    \item Instabilités de fréquence
    \item Génération de bruit d'intensité
    \item Comportement chaotique
\end{itemize}

\subsection{Solution identifiée}

La solution technique est clairement identifiée :

\begin{remark}
\textbf{Solution : Isolateurs optiques}
\begin{itemize}
    \item \textbf{Type requis :} Isolateurs Faraday à \SI{1550}{\nano\meter}
    \item \textbf{Isolation minimale :} \SI{30}{\decibel}
    \item \textbf{Timeline :} Commande prévue juillet 2025
    \item \textbf{Impact attendu :} Résolution complète du problème de rétroaction
\end{itemize}
\end{remark}

\section{Comparaison avec les objectifs initiaux}

\subsection{Tableau de bord des réalisations}

Le \cref{tab:objectifs_realises} synthétise l'état d'avancement par rapport aux objectifs initiaux :

\begin{table}[ht]
    \centering
    \caption{Bilan des objectifs vs réalisations}
    \label{tab:objectifs_realises}
    \begin{tabular}{p{0.6\textwidth}cc}
        \toprule
        \textbf{Objectif} & \textbf{Statut} & \textbf{Niveau} \\
        \midrule
        Construction spectromètre THz (0.3-5 THz) & ✅ & Partiel (60\%) \\
        Intégration REFIMEVE et peigne fréquence & ✅ & Complet \\
        Automatisation complète du système & ✅ & Complet \\
        Développement algorithmes optimisation & ✅ & Complet \\
        Caractérisation et diagnostic performances & ✅ & Complet \\
        \bottomrule
    \end{tabular}
\end{table}

\subsection{Réalisations dépassant les attentes}

Certains aspects du projet ont dépassé les attentes initiales :

\begin{itemize}
    \item \textbf{Innovation algorithmique TEC} : gain de 33\% non anticipé
    \item \textbf{Diagnostic technique expert} : identification précise de la limitation
    \item \textbf{Architecture logicielle modulaire} : réutilisabilité étendue
    \item \textbf{Documentation complète} : transfert de connaissances facilité
\end{itemize}

\section{Métrologie et traçabilité}

\subsection{Précision fréquentielle}

En l'absence de wavemètre disponible, l'étalonnage fréquentiel s'appuie sur :
\begin{itemize}
    \item \textbf{Données constructeur DFB} : précision \SI{1}{\GHz} typique
    \item \textbf{Validation ponctuelle} : comparaison avec peigne de fréquence
    \item \textbf{Cohérence interne} : validation par recoupement des modèles
\end{itemize}

\subsection{Stabilisation métrologique}

La stabilisation du peigne de fréquence présente des performances variables :
\begin{itemize}
    \item \textbf{Lock radiofréquence} : stable jour après jour
    \item \textbf{Lock optique REFIMEVE} : instabilités quotidiennes observées
    \item \textbf{Solution adoptée} : horloge rubidium + GPS en backup
\end{itemize}

\section{Bilan technique global}

\subsection{Système fonctionnel réalisé}

Le spectromètre développé constitue un système complet et fonctionnel :
\begin{itemize}
    \item Architecture intégrée opérationnelle
    \item Gamme THz étendue accessible
    \item Automatisation complète implémentée
    \item Innovations algorithmiques validées
\end{itemize}

\subsection{Limitation principale identifiée et solution}

La rétroaction optique, clairement identifiée comme limitation critique, dispose d'une solution technique éprouvée. L'installation des isolateurs optiques en juillet 2025 devrait permettre l'obtention de spectres de qualité spectroscopique.

\subsection{Contributions originales}

Ce stage a généré plusieurs contributions originales :
\begin{itemize}
    \item Algorithme d'optimisation TEC (84.7\% réduction commandes)
    \item Architecture logicielle modulaire réutilisable
    \item Diagnostic technique systématique
    \item Documentation complète pour transfert de connaissances
\end{itemize}

Le chapitre suivant replace ces résultats dans leur contexte scientifique plus large et examine les perspectives d'application et de développement futur.