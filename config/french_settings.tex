% =============================================================================
% CONFIGURATION FRANÇAISE
% Paramètres spécifiques à la typographie française
% =============================================================================

% -----------------------------------------------------------------------------
% CONFIGURATION DE BABEL
% -----------------------------------------------------------------------------
\frenchbsetup{
    StandardLayout=true,                % Mise en page standard
    ThinColonSpace=true,               % Espace fine avant les deux-points
    FrenchFootnotes=true,              % Notes de bas de page à la française
    AutoSpaceFootnotes=true,           % Espacement automatique notes
    IndentFirst=true                   % Indentation du premier paragraphe
}

% -----------------------------------------------------------------------------
% GUILLEMETS FRANÇAIS
% -----------------------------------------------------------------------------
\DeclareQuoteStyle{french}
    {\guillemotleft\space}
    {\space\guillemotright}
    {\textquotedblleft}
    {\textquotedblright}

\setquotestyle{french}

% -----------------------------------------------------------------------------
% NOMS FRANÇAIS POUR LES ÉLÉMENTS
% -----------------------------------------------------------------------------

% Redéfinition des noms de chapitres et sections
\addto\captionsfrench{
    \renewcommand{\contentsname}{Sommaire}
    \renewcommand{\listfigurename}{Liste des figures}
    \renewcommand{\listtablename}{Liste des tableaux}
    \renewcommand{\bibname}{Bibliographie}
    \renewcommand{\refname}{Références}
    \renewcommand{\indexname}{Index}
    \renewcommand{\figurename}{Figure}
    \renewcommand{\tablename}{Tableau}
    \renewcommand{\partname}{Partie}
    \renewcommand{\chaptername}{Chapitre}
    \renewcommand{\appendixname}{Annexe}
    \renewcommand{\abstractname}{Résumé}
}

% -----------------------------------------------------------------------------
% CONFIGURATION SIUNITX POUR LE FRANÇAIS
% -----------------------------------------------------------------------------
\sisetup{
    locale=FR,                         % Localisation française
    detect-weight=true,                % Détection automatique du poids de police
    detect-family=true,                % Détection automatique de la famille de police
    output-decimal-marker={,},         % Virgule comme séparateur décimal
    group-separator={\,},              % Espace fine comme séparateur de milliers
    inter-unit-product=\ensuremath{{}\cdot{}}, % Point médian entre unités
    per-mode=symbol,                   % Mode slash pour "par"
    bracket-numbers=false,             % Pas de parenthèses pour les nombres
    tight-spacing=true,                % Espacement serré
    space-before-unit=true,            % Espace avant l'unité
    unit-font-command=\mathrm,         % Police droite pour les unités
    number-unit-separator=\,,          % Espace fine entre nombre et unité
    range-phrase={~à~},                % "à" pour les intervalles
    range-units=single                 % Unité une seule fois dans un intervalle
}

% Unités personnalisées pour la spectroscopie THz
\DeclareSIUnit\THz{\tera\hertz}
\DeclareSIUnit\GHz{\giga\hertz}
\DeclareSIUnit\MHz{\mega\hertz}
\DeclareSIUnit\kHz{\kilo\hertz}
\DeclareSIUnit\mA{\milli\ampere}
\DeclareSIUnit\pW{\pico\watt}
\DeclareSIUnit\nW{\nano\watt}
\DeclareSIUnit\uW{\micro\watt}
\DeclareSIUnit\mW{\milli\watt}

% -----------------------------------------------------------------------------
% CONFIGURATION DES RÉFÉRENCES CROISÉES
% -----------------------------------------------------------------------------
\crefformat{chapter}{chapitre~#2#1#3}
\crefformat{section}{section~#2#1#3}
\crefformat{subsection}{section~#2#1#3}
\crefformat{figure}{figure~#2#1#3}
\crefformat{table}{tableau~#2#1#3}
\crefformat{equation}{équation~#2#1#3}

% Versions plurielles
\crefmultiformat{chapter}{chapitres~#2#1#3}{ et~#2#1#3}{, #2#1#3}{ et~#2#1#3}
\crefmultiformat{section}{sections~#2#1#3}{ et~#2#1#3}{, #2#1#3}{ et~#2#1#3}
\crefmultiformat{figure}{figures~#2#1#3}{ et~#2#1#3}{, #2#1#3}{ et~#2#1#3}
\crefmultiformat{table}{tableaux~#2#1#3}{ et~#2#1#3}{, #2#1#3}{ et~#2#1#3}
\crefmultiformat{equation}{équations~#2#1#3}{ et~#2#1#3}{, #2#1#3}{ et~#2#1#3}

% -----------------------------------------------------------------------------
% DATES EN FRANÇAIS
% -----------------------------------------------------------------------------
\renewcommand{\today}{\number\day~\ifcase\month\or
    janvier\or février\or mars\or avril\or mai\or juin\or
    juillet\or août\or septembre\or octobre\or novembre\or décembre\fi
    ~\number\year}

% -----------------------------------------------------------------------------
% MACROS FRANÇAISES UTILES
% -----------------------------------------------------------------------------

% Abréviations courantes
\newcommand{\cad}{c.-à-d.~}            % c'est-à-dire
\newcommand{\cf}{cf.~}                 % confer
\newcommand{\etc}{etc.}                % et caetera
\newcommand{\ie}{i.e.~}                % id est
\newcommand{\eg}{e.g.~}                % exempli gratia
\newcommand{\vs}{vs~}                  % versus

% Espaces typographiques françaises
\newcommand{\espacesfines}{\,}         % Espace fine
\newcommand{\espacessecables}{~}       % Espace sécable
\newcommand{\espacesinsecables}{\nobreak\,} % Espace insécable fine

% -----------------------------------------------------------------------------
% CORRECTION ORTHOGRAPHIQUE
% -----------------------------------------------------------------------------
% Mots composés avec trait d'union pour éviter les coupures malheureuses
\hyphenation{
    long-terme
    court-terme
    sous-système
    multi-fréquence
    auto-réglage
    co-superviseur
}