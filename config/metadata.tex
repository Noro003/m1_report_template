% =============================================================================
% MÉTADONNÉES DU RAPPORT
% À personnaliser selon votre stage
% =============================================================================

% -----------------------------------------------------------------------------
% INFORMATIONS ÉTUDIANT
% -----------------------------------------------------------------------------
\newcommand{\studentname}{Norayr MALKHASYAN}                    % Nom de l'étudiant
\newcommand{\studentnumber}{22109895}             % Numéro étudiant
\newcommand{\studentemail}{norayr.malkhasyan@universite-paris-saclay.fr}            % Email étudiant

% -----------------------------------------------------------------------------
% INFORMATIONS ACADÉMIQUES
% -----------------------------------------------------------------------------
\newcommand{\university}{Université Paris-Saclay}          % Université
\newcommand{\program}{Master 1 Physique et Applications}   % Formation
\newcommand{\academicyear}{2024-2025}                      % Année universitaire

% -----------------------------------------------------------------------------
% INFORMATIONS STAGE
% -----------------------------------------------------------------------------
\newcommand{\internshiptitle}{%
    Precision Molecular Spectroscopy in the THz range%
}                                                           % Titre du stage

\newcommand{\internshipsubtitle}{%
    Développement d'un spectromètre THz dans la gamme 0,3-5 THz%
}                                                           % Sous-titre

\newcommand{\internshipperiod}{14 avril 2025 -- 31 juillet 2025} % Période
\newcommand{\internshipduration}{3,5 mois}                 % Durée

% -----------------------------------------------------------------------------
% INFORMATIONS LABORATOIRE
% -----------------------------------------------------------------------------
\newcommand{\laboratory}{%
    Institut des Sciences Moléculaires d'Orsay (ISMO)%
}                                                           % Laboratoire

\newcommand{\laboratoryaddress}{%
    Université Paris-Saclay, CNRS\\
    Bâtiment 520, 91405 Orsay Cedex, France%
}                                                           % Adresse laboratoire

\newcommand{\labdirector}{[Directeur du laboratoire]}      % Directeur labo

% -----------------------------------------------------------------------------
% ENCADREMENT
% -----------------------------------------------------------------------------
\newcommand{\supervisor}{Olivier PIRALI}                   % Encadrant principal
\newcommand{\supervisortitle}{%
    Directeur de Recherche CNRS, ISMO%
}                                                           % Titre encadrant
\newcommand{\supervisoremail}{olivier.pirali@cnrs.fr}      % Email encadrant

\newcommand{\cosupervisor}{[Co-encadrant si applicable]}   % Co-encadrant
\newcommand{\cosupervisortitle}{[Titre co-encadrant]}      % Titre co-encadrant
\newcommand{\cosupervisoremail}{[email.co-encadrant@cnrs.fr]} % Email co-encadrant

% -----------------------------------------------------------------------------
% ÉQUIPE DE RECHERCHE
% -----------------------------------------------------------------------------
\newcommand{\researchteam}{%
    Systèmes Moléculaires, Astrophysique et Environnement (SYSTEMAE)%
}                                                           % Équipe de recherche

\newcommand{\teamleader}{Olivier PIRALI}                  % Chef d'équipe
\newcommand{\teamsize}{41 membres}                   % Taille équipe

% -----------------------------------------------------------------------------
% INFORMATIONS PROJET
% -----------------------------------------------------------------------------
\newcommand{\projectcontext}{%
    Développement de techniques spectroscopiques haute résolution 
    pour l'étude de molécules d'intérêt astrophysique%
}                                                           % Contexte projet

\newcommand{\thzrange}{0,558 -- 1,455 THz}                % Gamme THz couverte
\newcommand{\keyinnovation}{%
    Optimisation TEC avec réduction de 84,7\% des commandes redondantes%
}                                                           % Innovation clé

% -----------------------------------------------------------------------------
% OBJECTIFS PRINCIPAUX
% -----------------------------------------------------------------------------
\newcommand{\objectiveone}{%
    Construction d'un spectromètre THz dans la gamme 0,3-5 THz%
}
\newcommand{\objectivetwo}{%
    Intégration du réseau REFIMEVE et du peigne de fréquence optique%
}
\newcommand{\objectivethree}{%
    Automatisation complète du système de mesure%
}
\newcommand{\objectivefour}{%
    Développement d'algorithmes d'optimisation%
}
\newcommand{\objectivefive}{%
    Caractérisation et diagnostic des performances%
}

% -----------------------------------------------------------------------------
% RÉSULTATS PRINCIPAUX
% -----------------------------------------------------------------------------
\newcommand{\mainresultone}{%
    Système spectromètre THz opérationnel sur 896 GHz%
}
\newcommand{\mainresulttwo}{%
    Algorithme d'optimisation TEC économisant 33\% du temps de scan%
}
\newcommand{\mainresultthree}{%
    Identification de la rétroaction optique comme limitation principale%
}

% -----------------------------------------------------------------------------
% MOTS-CLÉS
% -----------------------------------------------------------------------------
\newcommand{\keywordsfr}{%
    spectroscopie THz, photomélange, lasers DFB, 
    automatisation, optimisation TEC, ISMO%
}                                                           % Mots-clés français

\newcommand{\keywordsen}{%
    THz spectroscopy, photomixing, DFB lasers, 
    automation, TEC optimization, ISMO%
}                                                           % Mots-clés anglais

% -----------------------------------------------------------------------------
% RÉSUMÉS
% -----------------------------------------------------------------------------
\newcommand{\resumefr}{%
    Ce rapport présente le développement d'un spectromètre THz sub-Doppler 
    par photomélange couvrant la gamme 0,558-1,455 THz. Le système, basé sur 
    le battement de deux lasers DFB, intègre le réseau REFIMEVE pour la 
    métrologie fréquentielle et un algorithme de mesure. Les principales réalisations incluent la 
    construction d'un système automatisé fonctionnel, l'identification de 
    la rétroaction optique comme limitation critique, et le développement 
    d'innovations algorithmiques transférables à d'autres systèmes.%
}

\newcommand{\resumeen}{%
    This report presents the development of a sub-Doppler THz spectrometer 
    using photomixing covering the 0.558-1.455 THz range. The system, based 
    on the beating of two DFB lasers, integrates the REFIMEVE network for 
    frequency metrology and a TEC optimization algorithm reducing scan time 
    by 33\%. Main achievements include building a functional automated system, 
    identifying optical feedback as a critical limitation, and developing 
    algorithmic innovations transferable to other systems.%
}

% -----------------------------------------------------------------------------
% INFORMATIONS DE COMPILATION
% -----------------------------------------------------------------------------
\newcommand{\compiledate}{\today}                          % Date de compilation
\newcommand{\reportversion}{1.0}                           % Version du rapport

% -----------------------------------------------------------------------------
% CONFIGURATION BIBLIOGRAPHIE
% -----------------------------------------------------------------------------
\addbibresource{backmatter/bibliography.bib}               % Fichier bibliographie