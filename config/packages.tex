% =============================================================================
% PACKAGES CONFIGURATION
% Packages essentiels pour rapport de physique M1
% =============================================================================

% -----------------------------------------------------------------------------
% ENCODAGE ET LANGUE
% -----------------------------------------------------------------------------
\usepackage[utf8]{inputenc}
\usepackage[T1]{fontenc}
\usepackage[french]{babel}
\usepackage{csquotes}

% -----------------------------------------------------------------------------
% TYPOGRAPHIE ET POLICES
% -----------------------------------------------------------------------------
\usepackage{lmodern}                    % Police Latin Modern
\usepackage{microtype}                  % Amélioration de la typographie
\usepackage{setspace}                   % Espacement des lignes

% -----------------------------------------------------------------------------
% MATHÉMATIQUES ET PHYSIQUE
% -----------------------------------------------------------------------------
\usepackage{amsmath}                    % Équations mathématiques avancées
\usepackage{amssymb}                    % Symboles mathématiques
\usepackage{amsfonts}                   % Polices mathématiques
\usepackage{mathtools}                  % Outils mathématiques étendus
\usepackage{siunitx}                    % Unités SI et notation scientifique
\usepackage{physics}                    % Notation physique (dérivées, etc.)
\usepackage{tensor}                     % Notation tensorielle
\usepackage{braket}                     % Notation de Dirac

% -----------------------------------------------------------------------------
% GRAPHIQUES ET FIGURES
% -----------------------------------------------------------------------------
\usepackage{graphicx}                   % Inclusion d'images
\usepackage{float}                      % Positionnement des figures
\usepackage{subcaption}                 % Sous-figures
\usepackage{wrapfig}                    % Figures avec texte autour
\usepackage{tikz}                       % Dessins vectoriels
\usepackage{pgfplots}                   % Graphiques scientifiques
\usepackage{circuitikz}                 % Schémas électriques/optiques

% Configuration pgfplots
\pgfplotsset{compat=1.18}

% -----------------------------------------------------------------------------
% TABLEAUX
% -----------------------------------------------------------------------------
\usepackage{array}                      % Tableaux avancés
\usepackage{tabularx}                   % Tableaux avec largeur fixe
\usepackage{longtable}                  % Tableaux sur plusieurs pages
\usepackage{booktabs}                   % Tableaux professionnels
\usepackage{multirow}                   % Cellules fusionnées verticalement
\usepackage{multicol}                   % Colonnes multiples

% -----------------------------------------------------------------------------
% LISTES ET ÉNUMÉRATIONS
% -----------------------------------------------------------------------------
\usepackage{enumitem}                   % Personnalisation des listes

% -----------------------------------------------------------------------------
% CODE ET ALGORITHMES
% -----------------------------------------------------------------------------
\usepackage{listings}                   % Code source
\usepackage{algorithm}                  % Algorithmes
\usepackage{algpseudocode}             % Pseudo-code

% -----------------------------------------------------------------------------
% COULEURS
% -----------------------------------------------------------------------------
\usepackage[dvipsnames,table]{xcolor}  % Couleurs étendues

% -----------------------------------------------------------------------------
% RÉFÉRENCES ET LIENS
% -----------------------------------------------------------------------------
\usepackage[
    colorlinks=true,
    linkcolor=NavyBlue,
    citecolor=OliveGreen,
    filecolor=BrickRed,
    urlcolor=Blue,
    bookmarks=true,
    bookmarksopen=true,
    pdftitle={Rapport de Stage M1 - Spectroscopie THz},
    pdfauthor={Étudiant M1},
    pdfsubject={Spectroscopie THz sub-Doppler par photomélange},
    pdfkeywords={THz, spectroscopie, photomélange, DFB, ISMO}
]{hyperref}

\usepackage{cleveref}                   % Références intelligentes
\usepackage{url}                        % URLs

% -----------------------------------------------------------------------------
% BIBLIOGRAPHIE
% -----------------------------------------------------------------------------
\usepackage[
    backend=biber,
    style=numeric,
    sorting=none,
    maxbibnames=10
]{biblatex}

% -----------------------------------------------------------------------------
% MISE EN PAGE ET HEADERS
% -----------------------------------------------------------------------------
\usepackage{geometry}                   % Marges et mise en page
\usepackage{fancyhdr}                   % En-têtes et pieds de page
\usepackage{titlesec}                   % Personnalisation des titres
\usepackage{titletoc}                   % Personnalisation de la table des matières

% -----------------------------------------------------------------------------
% OUTILS DIVERS
% -----------------------------------------------------------------------------
\usepackage{lipsum}                     % Texte de remplissage (à supprimer)
\usepackage{blindtext}                  % Texte de remplissage (à supprimer)
\usepackage{todonotes}                  % Notes TODO
\usepackage{comment}                    % Commentaires multilignes