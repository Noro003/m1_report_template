% =============================================================================
% CONFIGURATION PHYSIQUE
% Paramètres spécifiques à la physique et spectroscopie THz
% =============================================================================

% -----------------------------------------------------------------------------
% CONFIGURATION DU PACKAGE PHYSICS
% -----------------------------------------------------------------------------
% Le package physics est déjà chargé dans packages.tex

% Redéfinition de certaines commandes pour éviter les conflits
\renewcommand{\div}{\nabla\cdot}       % Divergence
\renewcommand{\curl}{\nabla\times}     % Rotationnel
\renewcommand{\grad}{\nabla}           % Gradient

% -----------------------------------------------------------------------------
% MACROS POUR LA SPECTROSCOPIE THZ
% -----------------------------------------------------------------------------

% Fréquences et longueurs d'onde
\newcommand{\freq}[1]{\ensuremath{f_{#1}}}                    % Fréquence
\newcommand{\freqbeat}{\ensuremath{f_{\text{beat}}}}          % Fréquence de battement
\newcommand{\freqTHz}{\ensuremath{f_{\text{THz}}}}            % Fréquence THz
\newcommand{\freqref}{\ensuremath{f_{\text{ref}}}}            % Fréquence de référence
\newcommand{\wavelength}[1]{\ensuremath{\lambda_{#1}}}        % Longueur d'onde

% Lasers DFB
\newcommand{\laserI}{\ensuremath{\text{Laser}_1}}             % Laser 1
\newcommand{\laserII}{\ensuremath{\text{Laser}_2}}            % Laser 2
\newcommand{\currentI}{\ensuremath{I_1}}                      % Courant laser 1
\newcommand{\currentII}{\ensuremath{I_2}}                     % Courant laser 2
\newcommand{\tempI}{\ensuremath{T_1}}                         % Température laser 1
\newcommand{\tempII}{\ensuremath{T_2}}                        % Température laser 2

% Puissances
\newcommand{\power}[1]{\ensuremath{P_{#1}}}                   % Puissance générale
\newcommand{\powerTHz}{\ensuremath{P_{\text{THz}}}}           % Puissance THz
\newcommand{\poweropt}{\ensuremath{P_{\text{opt}}}}           % Puissance optique
\newcommand{\powermeas}{\ensuremath{P_{\text{mes}}}}          % Puissance mesurée

% Contrôleur TEC (Temperature Electric Cooler)
\newcommand{\TEC}{\text{TEC}}                                 % TEC
\newcommand{\TECopt}{\text{TEC}_{\text{opt}}}                % TEC optimisé
\newcommand{\Tset}{\ensuremath{T_{\text{set}}}}              % Température de consigne
\newcommand{\Tmes}{\ensuremath{T_{\text{mes}}}}              % Température mesurée

% Photomélange
\newcommand{\photomix}{\text{photomélange}}                   % Photomélange
\newcommand{\beatfreq}{\ensuremath{\Delta f}}                 % Fréquence de battement

% -----------------------------------------------------------------------------
% MACROS POUR LES MESURES ET ERREURS
% -----------------------------------------------------------------------------

% Incertitudes
\newcommand{\uncertainty}[2]{\ensuremath{#1 \pm #2}}         % Valeur ± incertitude
\newcommand{\reluncertainty}[1]{\ensuremath{\pm #1\,\%}}     % Incertitude relative en %

% Statistiques
\newcommand{\mean}[1]{\ensuremath{\langle #1 \rangle}}       % Moyenne
\newcommand{\std}[1]{\ensuremath{\sigma_{#1}}}               % Écart-type
\newcommand{\var}[1]{\ensuremath{\sigma^2_{#1}}}             % Variance

% Rapports signal/bruit
\newcommand{\SNR}{\ensuremath{\text{SNR}}}                   % Signal-to-Noise Ratio
\newcommand{\RSB}{\ensuremath{\text{RSB}}}                   % Rapport Signal/Bruit

% -----------------------------------------------------------------------------
% MACROS POUR LES INSTRUMENTS
% -----------------------------------------------------------------------------

% Amplificateur lock-in
\newcommand{\lockin}{\text{lock-in}}                         % Lock-in
\newcommand{\Xsignal}{\ensuremath{X}}                        % Signal X
\newcommand{\Ysignal}{\ensuremath{Y}}                        % Signal Y
\newcommand{\Rsignal}{\ensuremath{R}}                        % Signal R (magnitude)
\newcommand{\phase}{\ensuremath{\phi}}                       % Phase

% Bolomètre
\newcommand{\bolometer}{\text{bolomètre}}                    % Bolomètre
\newcommand{\responsivity}{\ensuremath{\mathcal{R}}}         % Responsivité

% Peigne de fréquence
\newcommand{\comb}{\text{peigne}}                            % Peigne de fréquence
\newcommand{\frep}{\ensuremath{f_{\text{rep}}}}             % Fréquence de répétition
\newcommand{\fceo}{\ensuremath{f_{\text{ceo}}}}             % Carrier-envelope offset

% -----------------------------------------------------------------------------
% MACROS POUR LES MOLÉCULES
% -----------------------------------------------------------------------------

% Formules chimiques courantes en spectroscopie
\newcommand{\HdO}{\ce{H2O}}                                  % Eau
\newcommand{\NHtrois}{\ce{NH3}}                              % Ammoniac  
\newcommand{\CHtroisOH}{\ce{CH3OH}}                          % Méthanol
\newcommand{\NHdeux}{\ce{NH2}}                               % Radical NH2
\newcommand{\HtroisOplus}{\ce{H3O+}}                         % Ion hydronium

% Transitions moléculaires
\newcommand{\transition}[2]{\ensuremath{#1 \rightarrow #2}}  % Transition J'←J
\newcommand{\rotlevel}[1]{\ensuremath{J = #1}}              % Niveau rotationnel

% -----------------------------------------------------------------------------
% CONSTANTES PHYSIQUES UTILES
% -----------------------------------------------------------------------------

% Constantes fondamentales
\newcommand{\lightspeed}{\ensuremath{c}}                     % Vitesse de la lumière
\newcommand{\planck}{\ensuremath{h}}                         % Constante de Planck
\newcommand{\hbar}{\ensuremath{\hbar}}                       % Constante de Planck réduite
\newcommand{\boltzmann}{\ensuremath{k_B}}                    % Constante de Boltzmann

% Constantes typiques THz
\newcommand{\freqDoppler}{\ensuremath{\Delta f_{\text{D}}}}  % Largeur Doppler
\newcommand{\freqnat}{\ensuremath{\Gamma}}                   % Largeur naturelle

% -----------------------------------------------------------------------------
% ENVIRONNEMENTS POUR ÉQUATIONS IMPORTANTES
% -----------------------------------------------------------------------------

% Équation de battement
\newenvironment{beatequation}
{\begin{equation}\tag{Battement}}
{\end{equation}}

% Équation de puissance
\newenvironment{powerequation}
{\begin{equation}\tag{Puissance}}
{\end{equation}}

% Équation de fréquence DFB
\newenvironment{dfbequation}
{\begin{equation}\tag{DFB}}
{\end{equation}}

% -----------------------------------------------------------------------------
% MACROS POUR LES RÉFÉRENCES À DES ÉQUIPEMENTS SPÉCIFIQUES
% -----------------------------------------------------------------------------

% Équipements ISMO
\newcommand{\ISMO}{\text{ISMO}}                              % Institut des Sciences Moléculaires d'Orsay
\newcommand{\REFIMEVE}{\text{REFIMEVE}}                      % Réseau REFIMEVE
\newcommand{\Toptica}{\text{Toptica}}                        % Constructeur Toptica
\newcommand{\Arroyo}{\text{Arroyo}}                          % Contrôleur Arroyo

% Modèles d'instruments
\newcommand{\SR}[1]{\text{SR830}}                           % Stanford Research SR830
\newcommand{\DFCcore}{\text{DFC CORE+}}                     % Toptica DFC CORE+

% -----------------------------------------------------------------------------
% FORMATAGE DES DONNÉES TECHNIQUES
% -----------------------------------------------------------------------------

% Ports de communication
\newcommand{\comport}[1]{\texttt{COM#1}}                    % Port COM
\newcommand{\gpibport}[1]{\texttt{GPIB0::#1::INSTR}}       % Port GPIB

% Paramètres de scan
\newcommand{\scanparam}[2]{\texttt{#1 = #2}}               % Paramètre = valeur