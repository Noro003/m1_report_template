% Exemple de figure TikZ - Schéma système THz
% À utiliser dans vos chapitres avec % Exemple de figure TikZ - Schéma système THz
% À utiliser dans vos chapitres avec % Exemple de figure TikZ - Schéma système THz
% À utiliser dans vos chapitres avec % Exemple de figure TikZ - Schéma système THz
% À utiliser dans vos chapitres avec \input{figures/schema_systeme_exemple.tex}

\begin{tikzpicture}[
    % Styles des éléments
    laser/.style={rectangle, fill=red!20, draw=red!50, thick, minimum width=1.5cm, minimum height=0.8cm},
    device/.style={rectangle, fill=blue!20, draw=blue!50, thick, minimum width=1.2cm, minimum height=0.8cm},
    detector/.style={circle, fill=green!20, draw=green!50, thick, minimum size=1cm},
    beam/.style={->, thick, red},
    signal/.style={->, thick, blue, dashed}
]

% Lasers DFB
\node[laser] (laser1) at (0,2) {Laser 1\\DFB};
\node[laser] (laser2) at (0,0) {Laser 2\\DFB};

% Annotations fréquences
\node[above=0.2cm of laser1, font=\small] {$f_1 = \SI{192.0}{\THz}$};
\node[below=0.2cm of laser2, font=\small] {$f_2 = \SI{192.6}{\THz}$};

% Combineur optique
\node[device] (combiner) at (3,1) {Combineur};

% Photomélangeur
\node[device] (photomixer) at (6,1) {Photo-\\mélangeur};

% Source THz
\node[above=0.3cm of photomixer, font=\small, color=red] {$f_{THz} = |f_1 - f_2|$};
\node[above=0.6cm of photomixer, font=\small, color=red] {$= \SI{0.6}{\THz}$};

% Échantillon
\node[device] (sample) at (9,1) {Échantillon\\gazeux};

% Détecteur
\node[detector] (detector) at (12,1) {Bolomètre};

% Lock-in amplifier
\node[device] (lockin) at (12,-1.5) {Lock-in\\SR830};

% Faisceaux optiques
\draw[beam] (laser1.east) -- (combiner.north west);
\draw[beam] (laser2.east) -- (combiner.south west);
\draw[beam] (combiner.east) -- (photomixer.west);

% Faisceau THz
\draw[beam, orange, very thick] (photomixer.east) -- (sample.west);
\draw[beam, orange, very thick] (sample.east) -- (detector.west);

% Signaux électriques
\draw[signal] (detector.south) -- (lockin.north);

% Contrôleurs TEC
\node[device, fill=yellow!20, draw=yellow!50] (tec1) at (0,3.5) {TEC\\Controller};
\node[device, fill=yellow!20, draw=yellow!50] (tec2) at (0,-1.5) {TEC\\Controller};

\draw[signal] (tec1.south) -- (laser1.north);
\draw[signal] (tec2.north) -- (laser2.south);

% Annotations de contrôle
\node[left=0.5cm of tec1, font=\tiny] {$I_1, T_1$};
\node[left=0.5cm of tec2, font=\tiny] {$I_2, T_2$};

% Métrologie (peigne de fréquence)
\node[device, fill=purple!20, draw=purple!50] (comb) at (3,3) {Peigne\\fréquence};
\draw[signal, purple] (comb.south west) -- (laser1.north east);

% REFIMEVE
\node[device, fill=cyan!20, draw=cyan!50] (refimeve) at (6,3) {REFIMEVE};
\draw[signal, cyan] (refimeve.west) -- (comb.east);

% Système d'acquisition
\node[device, fill=gray!20, draw=gray!50] (computer) at (9,-1.5) {PC\\Contrôle};
\draw[signal] (lockin.west) -- (computer.east);
\draw[signal] (computer.north) -- (sample.south);

% Légendes
\begin{scope}[shift={(14,2)}]
    \node[font=\scriptsize] at (0,0.5) {\textbf{Légende:}};
    \draw[beam] (0,0) -- (0.5,0) node[right, font=\scriptsize] {Faisceau optique};
    \draw[beam, orange, very thick] (0,-0.3) -- (0.5,-0.3) node[right, font=\scriptsize] {Faisceau THz};
    \draw[signal] (0,-0.6) -- (0.5,-0.6) node[right, font=\scriptsize] {Signal électrique};
\end{scope}

\end{tikzpicture}

\begin{tikzpicture}[
    % Styles des éléments
    laser/.style={rectangle, fill=red!20, draw=red!50, thick, minimum width=1.5cm, minimum height=0.8cm},
    device/.style={rectangle, fill=blue!20, draw=blue!50, thick, minimum width=1.2cm, minimum height=0.8cm},
    detector/.style={circle, fill=green!20, draw=green!50, thick, minimum size=1cm},
    beam/.style={->, thick, red},
    signal/.style={->, thick, blue, dashed}
]

% Lasers DFB
\node[laser] (laser1) at (0,2) {Laser 1\\DFB};
\node[laser] (laser2) at (0,0) {Laser 2\\DFB};

% Annotations fréquences
\node[above=0.2cm of laser1, font=\small] {$f_1 = \SI{192.0}{\THz}$};
\node[below=0.2cm of laser2, font=\small] {$f_2 = \SI{192.6}{\THz}$};

% Combineur optique
\node[device] (combiner) at (3,1) {Combineur};

% Photomélangeur
\node[device] (photomixer) at (6,1) {Photo-\\mélangeur};

% Source THz
\node[above=0.3cm of photomixer, font=\small, color=red] {$f_{THz} = |f_1 - f_2|$};
\node[above=0.6cm of photomixer, font=\small, color=red] {$= \SI{0.6}{\THz}$};

% Échantillon
\node[device] (sample) at (9,1) {Échantillon\\gazeux};

% Détecteur
\node[detector] (detector) at (12,1) {Bolomètre};

% Lock-in amplifier
\node[device] (lockin) at (12,-1.5) {Lock-in\\SR830};

% Faisceaux optiques
\draw[beam] (laser1.east) -- (combiner.north west);
\draw[beam] (laser2.east) -- (combiner.south west);
\draw[beam] (combiner.east) -- (photomixer.west);

% Faisceau THz
\draw[beam, orange, very thick] (photomixer.east) -- (sample.west);
\draw[beam, orange, very thick] (sample.east) -- (detector.west);

% Signaux électriques
\draw[signal] (detector.south) -- (lockin.north);

% Contrôleurs TEC
\node[device, fill=yellow!20, draw=yellow!50] (tec1) at (0,3.5) {TEC\\Controller};
\node[device, fill=yellow!20, draw=yellow!50] (tec2) at (0,-1.5) {TEC\\Controller};

\draw[signal] (tec1.south) -- (laser1.north);
\draw[signal] (tec2.north) -- (laser2.south);

% Annotations de contrôle
\node[left=0.5cm of tec1, font=\tiny] {$I_1, T_1$};
\node[left=0.5cm of tec2, font=\tiny] {$I_2, T_2$};

% Métrologie (peigne de fréquence)
\node[device, fill=purple!20, draw=purple!50] (comb) at (3,3) {Peigne\\fréquence};
\draw[signal, purple] (comb.south west) -- (laser1.north east);

% REFIMEVE
\node[device, fill=cyan!20, draw=cyan!50] (refimeve) at (6,3) {REFIMEVE};
\draw[signal, cyan] (refimeve.west) -- (comb.east);

% Système d'acquisition
\node[device, fill=gray!20, draw=gray!50] (computer) at (9,-1.5) {PC\\Contrôle};
\draw[signal] (lockin.west) -- (computer.east);
\draw[signal] (computer.north) -- (sample.south);

% Légendes
\begin{scope}[shift={(14,2)}]
    \node[font=\scriptsize] at (0,0.5) {\textbf{Légende:}};
    \draw[beam] (0,0) -- (0.5,0) node[right, font=\scriptsize] {Faisceau optique};
    \draw[beam, orange, very thick] (0,-0.3) -- (0.5,-0.3) node[right, font=\scriptsize] {Faisceau THz};
    \draw[signal] (0,-0.6) -- (0.5,-0.6) node[right, font=\scriptsize] {Signal électrique};
\end{scope}

\end{tikzpicture}

\begin{tikzpicture}[
    % Styles des éléments
    laser/.style={rectangle, fill=red!20, draw=red!50, thick, minimum width=1.5cm, minimum height=0.8cm},
    device/.style={rectangle, fill=blue!20, draw=blue!50, thick, minimum width=1.2cm, minimum height=0.8cm},
    detector/.style={circle, fill=green!20, draw=green!50, thick, minimum size=1cm},
    beam/.style={->, thick, red},
    signal/.style={->, thick, blue, dashed}
]

% Lasers DFB
\node[laser] (laser1) at (0,2) {Laser 1\\DFB};
\node[laser] (laser2) at (0,0) {Laser 2\\DFB};

% Annotations fréquences
\node[above=0.2cm of laser1, font=\small] {$f_1 = \SI{192.0}{\THz}$};
\node[below=0.2cm of laser2, font=\small] {$f_2 = \SI{192.6}{\THz}$};

% Combineur optique
\node[device] (combiner) at (3,1) {Combineur};

% Photomélangeur
\node[device] (photomixer) at (6,1) {Photo-\\mélangeur};

% Source THz
\node[above=0.3cm of photomixer, font=\small, color=red] {$f_{THz} = |f_1 - f_2|$};
\node[above=0.6cm of photomixer, font=\small, color=red] {$= \SI{0.6}{\THz}$};

% Échantillon
\node[device] (sample) at (9,1) {Échantillon\\gazeux};

% Détecteur
\node[detector] (detector) at (12,1) {Bolomètre};

% Lock-in amplifier
\node[device] (lockin) at (12,-1.5) {Lock-in\\SR830};

% Faisceaux optiques
\draw[beam] (laser1.east) -- (combiner.north west);
\draw[beam] (laser2.east) -- (combiner.south west);
\draw[beam] (combiner.east) -- (photomixer.west);

% Faisceau THz
\draw[beam, orange, very thick] (photomixer.east) -- (sample.west);
\draw[beam, orange, very thick] (sample.east) -- (detector.west);

% Signaux électriques
\draw[signal] (detector.south) -- (lockin.north);

% Contrôleurs TEC
\node[device, fill=yellow!20, draw=yellow!50] (tec1) at (0,3.5) {TEC\\Controller};
\node[device, fill=yellow!20, draw=yellow!50] (tec2) at (0,-1.5) {TEC\\Controller};

\draw[signal] (tec1.south) -- (laser1.north);
\draw[signal] (tec2.north) -- (laser2.south);

% Annotations de contrôle
\node[left=0.5cm of tec1, font=\tiny] {$I_1, T_1$};
\node[left=0.5cm of tec2, font=\tiny] {$I_2, T_2$};

% Métrologie (peigne de fréquence)
\node[device, fill=purple!20, draw=purple!50] (comb) at (3,3) {Peigne\\fréquence};
\draw[signal, purple] (comb.south west) -- (laser1.north east);

% REFIMEVE
\node[device, fill=cyan!20, draw=cyan!50] (refimeve) at (6,3) {REFIMEVE};
\draw[signal, cyan] (refimeve.west) -- (comb.east);

% Système d'acquisition
\node[device, fill=gray!20, draw=gray!50] (computer) at (9,-1.5) {PC\\Contrôle};
\draw[signal] (lockin.west) -- (computer.east);
\draw[signal] (computer.north) -- (sample.south);

% Légendes
\begin{scope}[shift={(14,2)}]
    \node[font=\scriptsize] at (0,0.5) {\textbf{Légende:}};
    \draw[beam] (0,0) -- (0.5,0) node[right, font=\scriptsize] {Faisceau optique};
    \draw[beam, orange, very thick] (0,-0.3) -- (0.5,-0.3) node[right, font=\scriptsize] {Faisceau THz};
    \draw[signal] (0,-0.6) -- (0.5,-0.6) node[right, font=\scriptsize] {Signal électrique};
\end{scope}

\end{tikzpicture}

\begin{tikzpicture}[
    % Styles des éléments
    laser/.style={rectangle, fill=red!20, draw=red!50, thick, minimum width=1.5cm, minimum height=0.8cm},
    device/.style={rectangle, fill=blue!20, draw=blue!50, thick, minimum width=1.2cm, minimum height=0.8cm},
    detector/.style={circle, fill=green!20, draw=green!50, thick, minimum size=1cm},
    beam/.style={->, thick, red},
    signal/.style={->, thick, blue, dashed}
]

% Lasers DFB
\node[laser] (laser1) at (0,2) {Laser 1\\DFB};
\node[laser] (laser2) at (0,0) {Laser 2\\DFB};

% Annotations fréquences
\node[above=0.2cm of laser1, font=\small] {$f_1 = \SI{192.0}{\THz}$};
\node[below=0.2cm of laser2, font=\small] {$f_2 = \SI{192.6}{\THz}$};

% Combineur optique
\node[device] (combiner) at (3,1) {Combineur};

% Photomélangeur
\node[device] (photomixer) at (6,1) {Photo-\\mélangeur};

% Source THz
\node[above=0.3cm of photomixer, font=\small, color=red] {$f_{THz} = |f_1 - f_2|$};
\node[above=0.6cm of photomixer, font=\small, color=red] {$= \SI{0.6}{\THz}$};

% Échantillon
\node[device] (sample) at (9,1) {Échantillon\\gazeux};

% Détecteur
\node[detector] (detector) at (12,1) {Bolomètre};

% Lock-in amplifier
\node[device] (lockin) at (12,-1.5) {Lock-in\\SR830};

% Faisceaux optiques
\draw[beam] (laser1.east) -- (combiner.north west);
\draw[beam] (laser2.east) -- (combiner.south west);
\draw[beam] (combiner.east) -- (photomixer.west);

% Faisceau THz
\draw[beam, orange, very thick] (photomixer.east) -- (sample.west);
\draw[beam, orange, very thick] (sample.east) -- (detector.west);

% Signaux électriques
\draw[signal] (detector.south) -- (lockin.north);

% Contrôleurs TEC
\node[device, fill=yellow!20, draw=yellow!50] (tec1) at (0,3.5) {TEC\\Controller};
\node[device, fill=yellow!20, draw=yellow!50] (tec2) at (0,-1.5) {TEC\\Controller};

\draw[signal] (tec1.south) -- (laser1.north);
\draw[signal] (tec2.north) -- (laser2.south);

% Annotations de contrôle
\node[left=0.5cm of tec1, font=\tiny] {$I_1, T_1$};
\node[left=0.5cm of tec2, font=\tiny] {$I_2, T_2$};

% Métrologie (peigne de fréquence)
\node[device, fill=purple!20, draw=purple!50] (comb) at (3,3) {Peigne\\fréquence};
\draw[signal, purple] (comb.south west) -- (laser1.north east);

% REFIMEVE
\node[device, fill=cyan!20, draw=cyan!50] (refimeve) at (6,3) {REFIMEVE};
\draw[signal, cyan] (refimeve.west) -- (comb.east);

% Système d'acquisition
\node[device, fill=gray!20, draw=gray!50] (computer) at (9,-1.5) {PC\\Contrôle};
\draw[signal] (lockin.west) -- (computer.east);
\draw[signal] (computer.north) -- (sample.south);

% Légendes
\begin{scope}[shift={(14,2)}]
    \node[font=\scriptsize] at (0,0.5) {\textbf{Légende:}};
    \draw[beam] (0,0) -- (0.5,0) node[right, font=\scriptsize] {Faisceau optique};
    \draw[beam, orange, very thick] (0,-0.3) -- (0.5,-0.3) node[right, font=\scriptsize] {Faisceau THz};
    \draw[signal] (0,-0.6) -- (0.5,-0.6) node[right, font=\scriptsize] {Signal électrique};
\end{scope}

\end{tikzpicture}