% =============================================================================
% ORGANISATION DU RAPPORT
% =============================================================================

\chapter*{Organisation du rapport}
\addcontentsline{toc}{chapter}{Organisation du rapport}

Ce rapport de stage de Master 1 s'organise en cinq chapitres principaux, suivant une structure logique qui retrace le développement complet du spectromètre THz par photomélange.

\section*{Structure du document}

\subsection*{Chapitre 1 : Introduction et contexte (4-5 pages)}
Ce chapitre présente le contexte scientifique et institutionnel du stage. Il débute par une présentation de l'Institut des Sciences Moléculaires d'Orsay (ISMO) et de l'équipe d'accueil, puis développe le contexte scientifique de la spectroscopie THz. Les objectifs du stage sont clairement énoncés, ainsi que les défis techniques spécifiques à relever dans la gamme de fréquences visée (> 1,5 THz).

\subsection*{Chapitre 2 : Protocoles et procédures (8-10 pages)}
Ce chapitre constitue le cœur technique du rapport. Il détaille la théorie du photomélange, l'architecture du système développé, et les protocoles expérimentaux mis en place. Une attention particulière est portée aux innovations algorithmiques, notamment l'optimisation du contrôle TEC qui constitue une contribution majeure de ce travail.

\subsection*{Chapitre 3 : Résultats vs attendus (6-8 pages)}
Ce chapitre analyse les performances obtenues en les confrontant aux spécifications initiales. Il présente les mesures spectroscopiques préliminaires, quantifie les gains de performance de l'optimisation TEC, et établit un diagnostic technique de la limitation principale identifiée (rétroaction optique).

\subsection*{Chapitre 4 : Analyse et perspectives (4-5 pages)}
Ce chapitre replace les résultats dans leur contexte scientifique plus large. Il évalue le potentiel du système pour des applications futures, notamment en astrophysique et physique fondamentale, et propose des pistes d'amélioration et d'extension du système.

\subsection*{Chapitre 5 : Conclusion générale (2-3 pages)}
La conclusion synthétise les accomplissements du stage, évalue l'apport personnel de l'étudiant au projet, et établit les liens avec la formation Master 1 Physique et Applications.

\section*{Éléments transversaux}

\subsection*{Approche méthodologique}
Le rapport suit une démarche scientifique rigoureuse, alliant théorie et expérimentation. Chaque développement technique est justifié par les contraintes physiques et les objectifs de performance.

\subsection*{Innovation technique}
L'accent est mis sur les innovations développées pendant le stage, particulièrement l'algorithme d'optimisation TEC qui représente une contribution originale au domaine.

\subsection*{Transfert de connaissances}
Les solutions développées sont présentées de manière à faciliter leur réutilisation dans d'autres contextes expérimentaux.

\newpage